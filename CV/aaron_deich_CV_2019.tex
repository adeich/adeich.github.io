% Adapted from layout by GaelVaroquaux
%  http://gael-varoquaux.info
%

\documentclass{article} %{{{--

\usepackage[paper=a4paper,
	    top=2cm,
	    left=1.35cm,
	    width=18.2cm,
	    bottom=2cm
	    ]{geometry}
            %margin=4cm,

\usepackage{calc}
\usepackage[T1]{fontenc}
\usepackage[utf8]{inputenc}
\usepackage{lmodern}
\usepackage{color,hyperref}
\usepackage{graphicx}
\usepackage{multicol}

\usepackage{wasysym} % For phone symbol
\usepackage{url}

\def\bf{\bfseries}
\def\sf{\sffamily}
\def\sl{\slshape}
% Semi condensed bold

\definecolor{deep_blue}{rgb}{0,.2,.5}
\definecolor{dark_blue}{rgb}{0,.1,.3}
\definecolor{myblue}{rgb}{.01,0.21,0.71}
\definecolor{gray}{rgb}{.5, .5, .5}

\hypersetup{pdftex,  % needed for pdflatex
  breaklinks=true,  % so long urls are correctly broken across lines
  colorlinks=true,
  urlcolor=myblue,
  %linkcolor=darkblue,
  %citecolor=darkgreen,
  }


%% This gives us fun enumeration environments.
\usepackage{enumitem}

%% More layout: Get rid of indenting throughout entire document
\setlength{\parindent}{0in}

%% Reference the last page in the page number
%
\usepackage{fancyhdr,lastpage}
\pagestyle{fancy}
%\pagestyle{empty}      % Uncomment this to get rid of page numbers
\fancyhf{}\renewcommand{\headrulewidth}{0pt}
%\fancyfootoffset{\marginparsep}+\marginparwidth}
\lfoot{
  \hspace{-2\marginparsep}
  \,\hfill \arabic{page} of \protect\pageref*{LastPage}  \hfill \,\\
  \,\hfill {\footnotesize \textcolor{gray}{updated July 2018~~~~~}} \hfill\,
}

\newcommand{\mydate}[1]{{\textcolor{gray}{\footnotesize #1}}}


\newcommand{\makeheading}[1]%
        {%\hspace*{-\marginparsep minus \marginparwidth}%
         %\begin{minipage}[t]{\textwidth+\marginparwidth+\marginparsep}%
         \begin{minipage}[t]{\textwidth}%
                {\Large #1}\\%[-0.5\baselineskip]%
                \vskip 0.2\baselineskip
                 \color{deep_blue}{\rule{\columnwidth}{3pt}}%
         \end{minipage}
	 \vskip 1.\baselineskip plus 2\baselineskip minus 1.\baselineskip
	}

\newlength\sidebarwidth
\setlength\sidebarwidth{3.6cm}

\newcommand{\topic}[3][]%
	 {\pagebreak[2]%
	 \vskip 1.5\baselineskip plus 3\baselineskip minus 0.7\baselineskip
	 \begin{minipage}{\textwidth}
         \phantomsection\addcontentsline{toc}{section}{#1}%
         \nopagebreak\hspace{0in}%
         \nopagebreak\begin{minipage}[t]{\sidebarwidth - .2cm}
         \raggedleft \bf\sf
	 \color{deep_blue}{\Large #2}
	 \end{minipage}%
	 \hfill
	 \begin{minipage}[t]{\linewidth - \sidebarwidth}
	 \nopagebreak{\color{deep_blue}%
		    \rule{0pt}{\baselineskip}%
		    \rule{\linewidth}{2.5pt}%
		    \llap{\raisebox{.3\baselineskip}{\sf #1}}%
		    \vspace*{.1\baselineskip}%
		    }%
	 #3%
	 \end{minipage}
	 \end{minipage}}

\newcommand{\smalltopic}[2]%
	 {\pagebreak[2]%
	 \vskip 1\baselineskip plus 2\baselineskip minus 0.3\baselineskip
	 \begin{minipage}{\textwidth}
	 %\hspace{-\marginparsep minus \marginparwidth}%
         \phantomsection\addcontentsline{toc}{subsection}{#1}%
         \nopagebreak\hspace{0in}%
         \nopagebreak\begin{minipage}[t]{\sidebarwidth - .2cm}
         \raggedleft \bf\sf %\vskip -0.5\baselineskip
	 \textcolor{dark_blue}{\large #1}%
	 \end{minipage}%
	 \hfill
	 \begin{minipage}[t]{\linewidth - \sidebarwidth}
	 \nopagebreak{%
	    %\vspace{-.7\baselineskip}%
	    \rule{\linewidth}{.5pt}%
	    \vspace{.1\baselineskip}%
	    }%
	    #2
	 \end{minipage}
	 \end{minipage}}

\newcommand{\subtopic}[3][]
	 {\begin{minipage}{\textwidth}
	 \vspace*{.4\baselineskip}
         \nopagebreak\hspace{0in}%
         \nopagebreak\begin{minipage}[t]{\sidebarwidth - .2cm}
	 % Super posh: using semi-bold condensed fonts. Works only with
	 % lmodern
         \raggedleft {\sf\fontseries{sbc}\selectfont #2}
	 %{\small\sl\\[-0.2\baselineskip] #1}
         {\\[-0.2\baselineskip] \textcolor{gray}{\footnotesize #1}}
	 \end{minipage}%
	 \hfill
	 \begin{minipage}[t]{\linewidth - \sidebarwidth}
	 #3%
	 \end{minipage}%
	 \vspace*{.2\baselineskip plus 1\baselineskip minus
	 .2\baselineskip}%
	 \end{minipage}}

\newcommand{\dateonly}[2][]
	 {\begin{minipage}{\textwidth}
	 \vspace*{.4\baselineskip}
         \nopagebreak\hspace{0in}%
         \nopagebreak\begin{minipage}[t]{\sidebarwidth - .2cm}
         \raggedleft {~}
         {\\[-\baselineskip] \textcolor{gray}{\footnotesize #1}}
	 \end{minipage}%
	 \hfill
	 \begin{minipage}[t]{\linewidth - \sidebarwidth}
	 #2%
	 \end{minipage}%
	 \vspace*{.2\baselineskip plus 1\baselineskip minus
	 .2\baselineskip}%
	 \end{minipage}}

\newcommand{\sidenote}[2]
	 {\vspace*{-.2\baselineskip}\begin{minipage}{\textwidth}
         \nopagebreak\hspace{0in}%
         \nopagebreak\begin{minipage}[t]{\sidebarwidth - .2cm}
         \raggedleft {#1}
	 \end{minipage}%
	 \hfill
	 \begin{minipage}[t]{\linewidth - \sidebarwidth}
	 #2%
	 \end{minipage}%
	 \vspace*{.5\baselineskip plus 1\baselineskip minus
	 .2\baselineskip}%
	 \end{minipage}}

% New lists environments
\newlist{outerlist}{itemize}{1}
\setlist[outerlist]{font=\sffamily\bfseries, label=\textbullet}
\setitemize{topsep=0ex, partopsep=0ex}
\setdescription{font=\normalfont\sffamily\bfseries, itemsep=.5ex,
    parsep=.5ex, leftmargin=3ex}

\newcommand{\blankline}{\quad\pagebreak[2]}

\def\mydot{\textcolor{deep_blue}{\rule{1ex}{1ex}}}

%%%%%%%%%%%%%%%%%%%%%%%%%%%%%%%%%%%%%%%%%%%%%%%%%%%%%%%%%%%%%%%%%%%%%--}}}%
\begin{document}
\makeheading{
\begin{minipage}[B]{0.5\textwidth}
    %\vfill
    \vspace*{-.5\baselineskip}%
    \parbox{10cm}{
	\hskip -0.1cm
	{\Huge\bf\sf \color{deep_blue} A\huge \hskip -0.05cm ARON %
	 \Huge D\huge \hskip -0.05cm EICH}
	\\[-.1\baselineskip]
	{\bf\sf Physicist \&}
        \\
        {\bf\sf Data Scientist}
       }
\end{minipage}
\hfill
\begin{minipage}[B]{8cm}
    \raggedleft
    \,\vskip -1em
    \small
	{\texttt {deichaaron@gmail.com}\\
    \vspace*{.5\baselineskip}%
         San Francisco, CA}%
    \vspace*{-.5\baselineskip}%
\end{minipage}
}

%\begin{center}
%\begin{minipage}{15cm}

\begin{center}
My driving goal is to use, support, and advocate open \& reproducible
software practices across disciplines.
\end{center}

\begin{multicols}{2}
\sloppy

%\textcolor{deep_blue}{\bf\sf Research interests}:
%Cosmology, weak lensing, data mining and automated learning algorithms for
%large astronomical data sets.

\begin{itemize}[leftmargin=2ex, itemsep=0ex]
\item[\mydot]
I am the Director of Open Software at the University of Washington's eScience
institute, an interdisciplinary program designed to support data-driven
discovery across disciplines. At the institute I work with students,
researchers, and faculty in a variety of settings.

\item[\mydot]
I maintain a technical blog, \href{http://jakevdp.github.io/}{Pythonic Perambulations}, to share tutorials and opinions related to statistics, open software, and scientific computing in Python.

\item[\mydot]
I invest a significant amount of time in creating and developing
Python tools for use in data-intensive science, including packages like
{\it Scikit-Learn}, {\it SciPy}, {\it AstroPy}, {\it Altair}, and many others.

\smalltopic{Software}{}

   \subtopic[2010--Present]{Scikit-Learn}{
     I a member of the core team of
     \href{http://scikit-learn.org}{scikit-learn},
     a popular package for performing machine learning in Python.  I
     have contributed in many areas, but most notably routines for efficient
     2-point (e.g. nearest neighbors) queries, and algorithms based on these
     such as {\it k}-neighbor classification, kernel density estimation,
     and manifold learning.  I have also presented tutorials on the subject
     on many occasions, including at the PyCon, SciPy, and PyData
     conferences.}

   \subtopic[2011--Present]{SciPy}{
     I am a maintainer of
     \href{http://scipy.org}{SciPy}, the definitive repository for many
     scientific computing tools available in Python.
     My contributions are primarily in the sparse
     matrix package, including code for efficient solutions of large sparse
     eigenvalue problems, and for efficient traversal and analysis of
     large sparse graphs.
   }

   \subtopic[2016-Present]{Altair}{
     I am co-creator of the \href{http://github.com/ellisonbg/altair}{Altair}
     project, a declarative statistical visualization library for Python built
     on the Vega-Lite visualization grammar.
   }

   \subtopic[2014--Present]{AstroPy}{
	 I have contributed several components of the suite of statistics tools
	 for the \href{http://astropy.org}{AstroPy} project, a Python package aimed at astronomers. In
	 particular, I wrote the modules for \href{http://docs.astropy.org/en/stable/visualization/histogram.html}{Bayesian Blocks}
	 and the \href{http://docs.astropy.org/en/stable/stats/lombscargle.html}{Lomb-Scargle Periodogram}.
   }

   \subtopic[]{Others}{
     I have created and contributed to many other Python projects, including
     Matplotlib, IPython, NumPy, Pandas, AstroML, SciDB-Py, Pelican, mpld3,
     and others.
     I have also open-sourced much of my research code and teaching materials.
     More information is available in my
     \href{http://github.com/jakevdp}{GitHub profile}.
   }

\end{document}