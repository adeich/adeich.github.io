% Adapted from layout by GaelVaroquaux
%  http://gael-varoquaux.info
%

\documentclass{article} %{{{--

\usepackage[paper=a4paper,
	    top=2cm,
	    left=1.35cm,
	    width=18.2cm,
	    bottom=2cm
	    ]{geometry}
            %margin=4cm,

\usepackage{calc}
\usepackage[T1]{fontenc}
\usepackage[utf8]{inputenc}
\usepackage{lmodern}
\usepackage{color,hyperref}
\usepackage{graphicx}
\usepackage{multicol}

\usepackage{wasysym} % For phone symbol
\usepackage{url}

\def\bf{\bfseries}
\def\sf{\sffamily}
\def\sl{\slshape}
% Semi condensed bold

\definecolor{deep_blue}{rgb}{0,.2,.5}
\definecolor{dark_blue}{rgb}{0,.1,.3}
\definecolor{myblue}{rgb}{.01,0.21,0.71}
\definecolor{gray}{rgb}{.5, .5, .5}

\hypersetup{pdftex,  % needed for pdflatex
  breaklinks=true,  % so long urls are correctly broken across lines
  colorlinks=true,
  urlcolor=myblue,
  %linkcolor=darkblue,
  %citecolor=darkgreen,
  }


%% This gives us fun enumeration environments.
\usepackage{enumitem}

%% More layout: Get rid of indenting throughout entire document
\setlength{\parindent}{0in}

%% Reference the last page in the page number
%
\usepackage{fancyhdr,lastpage}
\pagestyle{fancy}
%\pagestyle{empty}      % Uncomment this to get rid of page numbers
\fancyhf{}\renewcommand{\headrulewidth}{0pt}
%\fancyfootoffset{\marginparsep}+\marginparwidth}
\lfoot{
  \hspace{-2\marginparsep}
  \,\hfill \arabic{page} of \protect\pageref*{LastPage}  \hfill \,\\
  \,\hfill {\footnotesize \textcolor{gray}{updated \today~~~~~}} \hfill\,
}

\newcommand{\mydate}[1]{{\textcolor{gray}{\footnotesize #1}}}


\newcommand{\makeheading}[1]%
        {%\hspace*{-\marginparsep minus \marginparwidth}%
         %\begin{minipage}[t]{\textwidth+\marginparwidth+\marginparsep}%
         \begin{minipage}[t]{\textwidth}%
                {\Large #1}\\%[-0.5\baselineskip]%
                \vskip 0.2\baselineskip
                 \color{deep_blue}{\rule{\columnwidth}{3pt}}%
         \end{minipage}
	 \vskip 1.\baselineskip plus 2\baselineskip minus 1.\baselineskip
	}

\newlength\sidebarwidth
\setlength\sidebarwidth{3.6cm}

\newcommand{\topic}[3][]%
	 {\pagebreak[2]%
	 \vskip 1.5\baselineskip plus 3\baselineskip minus 0.7\baselineskip
	 \begin{minipage}{\textwidth}
         \phantomsection\addcontentsline{toc}{section}{#1}%
         \nopagebreak\hspace{0in}%
         \nopagebreak\begin{minipage}[t]{\sidebarwidth - .2cm}
         \raggedleft \bf\sf
	 \color{deep_blue}{\Large #2}
	 \end{minipage}%
	 \hfill
	 \begin{minipage}[t]{\linewidth - \sidebarwidth}
	 \nopagebreak{\color{deep_blue}%
		    \rule{0pt}{\baselineskip}%
		    \rule{\linewidth}{2.5pt}%
		    \llap{\raisebox{.3\baselineskip}{\sf #1}}%
		    \vspace*{.1\baselineskip}%
		    }%
	 #3%
	 \end{minipage}
	 \end{minipage}}

\newcommand{\smalltopic}[2]%
	 {\pagebreak[2]%
	 \vskip 1\baselineskip plus 2\baselineskip minus 0.3\baselineskip
	 \begin{minipage}{\textwidth}
	 %\hspace{-\marginparsep minus \marginparwidth}%
         \phantomsection\addcontentsline{toc}{subsection}{#1}%
         \nopagebreak\hspace{0in}%
         \nopagebreak\begin{minipage}[t]{\sidebarwidth - .2cm}
         \raggedleft \bf\sf %\vskip -0.5\baselineskip
	 \textcolor{dark_blue}{\large #1}%
	 \end{minipage}%
	 \hfill
	 \begin{minipage}[t]{\linewidth - \sidebarwidth}
	 \nopagebreak{%
	    %\vspace{-.7\baselineskip}%
	    \rule{\linewidth}{.5pt}%
	    \vspace{.1\baselineskip}%
	    }%
	    #2
	 \end{minipage}
	 \end{minipage}}

\newcommand{\subtopic}[3][]
	 {\begin{minipage}{\textwidth}
	 \vspace*{.4\baselineskip}
         \nopagebreak\hspace{0in}%
         \nopagebreak\begin{minipage}[t]{\sidebarwidth - .2cm}
	 % Super posh: using semi-bold condensed fonts. Works only with
	 % lmodern
         \raggedleft {\sf\fontseries{sbc}\selectfont #2}
	 %{\small\sl\\[-0.2\baselineskip] #1}
         {\\[-0.2\baselineskip] \textcolor{gray}{\footnotesize #1}}
	 \end{minipage}%
	 \hfill
	 \begin{minipage}[t]{\linewidth - \sidebarwidth}
	 #3%
	 \end{minipage}%
	 \vspace*{.2\baselineskip plus 1\baselineskip minus
	 .2\baselineskip}%
	 \end{minipage}}

\newcommand{\dateonly}[2][]
	 {\begin{minipage}{\textwidth}
	 \vspace*{.4\baselineskip}
         \nopagebreak\hspace{0in}%
         \nopagebreak\begin{minipage}[t]{\sidebarwidth - .2cm}
         \raggedleft {~}
         {\\[-\baselineskip] \textcolor{gray}{\footnotesize #1}}
	 \end{minipage}%
	 \hfill
	 \begin{minipage}[t]{\linewidth - \sidebarwidth}
	 #2%
	 \end{minipage}%
	 \vspace*{.2\baselineskip plus 1\baselineskip minus
	 .2\baselineskip}%
	 \end{minipage}}

\newcommand{\sidenote}[2]
	 {\vspace*{-.2\baselineskip}\begin{minipage}{\textwidth}
         \nopagebreak\hspace{0in}%
         \nopagebreak\begin{minipage}[t]{\sidebarwidth - .2cm}
         \raggedleft {#1}
	 \end{minipage}%
	 \hfill
	 \begin{minipage}[t]{\linewidth - \sidebarwidth}
	 #2%
	 \end{minipage}%
	 \vspace*{.5\baselineskip plus 1\baselineskip minus
	 .2\baselineskip}%
	 \end{minipage}}

% New lists environments
\newlist{outerlist}{itemize}{1}
\setlist[outerlist]{font=\sffamily\bfseries, label=\textbullet}
\setitemize{topsep=0ex, partopsep=0ex}
\setdescription{font=\normalfont\sffamily\bfseries, itemsep=.5ex,
    parsep=.5ex, leftmargin=3ex}

\newcommand{\blankline}{\quad\pagebreak[2]}

\def\mydot{\textcolor{deep_blue}{\rule{1ex}{1ex}}}

%%%%%%%%%%%%%%%%%%%%%%%%%%%%%%%%%%%%%%%%%%%%%%%%%%%%%%%%%%%%%%%%%%%%%--}}}%
\begin{document}
\makeheading{
\begin{minipage}[B]{0.5\textwidth}
    %\vfill
    \vspace*{-.5\baselineskip}%
    \parbox{10cm}{
	\hskip -0.1cm
	{\Huge\bf\sf \color{deep_blue} A\huge \hskip -0.05cm ARON %
	 \Huge D\huge \hskip -0.05cm EICH}
	\\[-.1\baselineskip]
	{\bf\sf Physicist \&}
        \\
        {\bf\sf Data Scientist}
       }
\end{minipage}
\hfill
\begin{minipage}[B]{8cm}
    \raggedleft
    \,\vskip -1em
    \small
	{\texttt {deichaaron(at)gmail.com}\\
    \vspace*{.5\baselineskip}%
         San Francisco, CA}%
    \vspace*{-.5\baselineskip}%
\end{minipage}
}

%\begin{center}
%\begin{minipage}{15cm}

%\begin{center}
%Here is my little about-me pitch
%\end{center}

\begin{multicols}{2}
\sloppy

%\textcolor{deep_blue}{\bf\sf Research interests}:
%Cosmology, weak lensing, data mining and automated learning algorithms for
%large astronomical data sets.

\begin{itemize}[leftmargin=2ex, itemsep=0ex]
\item[\mydot]
With a background in physics and scientific computation, I have broad experience in applying computational and statistical techniques to describe patterns within data, and in creating models to explain and predict these patterns. I am highly effective in learning how systems work and gaining insight into their behavior.
\item[\mydot]
Additionally, with my significant industry software engineering and computer science experience, I am proficient at writing readable, robust, code, in collaboration with software teams. I have strong familiarity with both object-oriented and functional programming paradigms.

\item[\mydot]
I have recently been a physics and calculus high school teacher. Ready to begin applying the math and analytical tools I was teaching towards original, creative projects, I have switched to data science full time. 
\end{itemize}
\end{multicols}
\vspace*{-1.5em}
\fussy

%%%%%%%%%%%%%%%%%%%%%%%%%%%%%%%%%%%%%%%%%%%%%%%%%%%%%%%%%%%%%%%%%%%%%%

\smalltopic{Data Science and Software Experience}{}

	\subtopic[2018-2019]{Lick Observatory}{
	I ranked causes of unexpectedly high starlight measurement error for an
	 \$8M robotic planet-hunting telescope, describing correlation and dependency across dozens of telescope sensor channels (temperature, windspeed, etc). I combined millions of records across mismatched SQL telemetry databases and used Jupyter Python notebooks to perform regression and PCA analysis to look for causal relationships. Through frequent discussion with the telescope engineers, my insights were instrumental in guiding where to focus improvement efforts, and the observatory has since reduced the telescope’s error to theoretical minimum levels. Github project }
	 
	 \subtopic[2013-2014]{Independent Python Developer}{
	 I developed DNA sequence string pattern-matching Python objects for use by a cancer research lab. My scripts allowed the lab's scientists to search for arbitrary sets of DNA strings from a high-level API.
	 }
	
	\subtopic[2012-2013]{Markit}{
	To support Markit’s growth of serving several new financial institutions per month, I built and maintained dozens of custom ETL pipelines for ingesting and storing financial data from remote servers. I wrote Python for high-frequency file transfer and transformation; SQL scripts to maintain error-free behavior in our production environment; and logic to manage robust performance across our three local, redundant, company data centers. 

	} 
%%%%%%%%%%%%%%%%%%%%%%%%%%%%%%%%%%%%%%%%%%%%%%%%%%%%
\topic{T \large\hskip -1ex echnical Language Skills and Expertise}{~}

\subtopic[~]{~}{

	\begin{itemize}
	\item Python---\texttt{ numpy, scipy, matplotlib, pandas, scikit-learn}
	\item SQL, C, Bash, \LaTeX, Mathematica
	\item Statistics: Hypothesis testing, bayesian analysis, regressions, A/B testing, Monte Carlo simulation.
	\item Mathematics: multivariate calculus, numerical integration, differential Eqn solving	, analytical modeling, linear algebra
	\item Algorithms: Dimensionality reduction (PCA), fourier analysis, clustering and classification ML.
	\end{itemize}
}


%%%%%%%%%%%%%%%%%%%%%%%%%%%%%%%%%%%%%%%%%%%%%%%%%%%%
\topic{E \large\hskip -1ex ducation}{~}

\subtopic[2018]{San Jose State University}{
     MA in Physics; focus on scientific computation and data analysis techniques spanning linear algebra, signal processing, machine learning, numerical integration, etc. 
     
     \textit{Masters Thesis:} I made a study looking at why people don't read physics textbooks for fun; how textbooks today are fundamentally bad at teaching people to be good at creative, original understanding; and I looked at what we can learn from the books that we do read for pleasure. 
   }



   \subtopic[2011]{Reed College}{
     BA in Physics; focus on computer science and computational physics. 
     
     \textit{Thesis}: I simulated the orbital effect of supernovae on orbital systems using a Monte Carlo simulation. } 
    

  %%%%%%%%%%%%%%%%%%%%%%%%%%%%%%%%%%%%%%%%%%%%%%%%%%%%

	

  
 %%%%%%%%%%%%%%%%%%%%%%%%%%%%%%%%%%%%%%%%%%%%%%%%%%%%
 
 \topic{T \large\hskip -1ex eaching Experience}{~}
 

\subtopic[2018-2019]{Jewish Community High School of the Bay}{
	AP Physics and AP Calculus Teacher; Taught 51 students for the year. 
	}

\subtopic[2016-2017]{Pacific Collegiate School}{
	Physics Teacher
	Taught 55 students; I developed all of my own material for teaching AP Physics and Conceptual Physics. 
 }
  \subtopic[2014-2018]{San Jose State University}{ 
 Teaching Assistant \\ 
 \begin{itemize}
 	\item Taught calculus-based Mechanics (Physics 50)
 	\item Physics for non-science majors (2A)
 	\item Physics of Music. Co-taught and designed lessons. 
 \end{itemize}
}

  %%%%%%%%%%%%%%%%%%%%%%%%%%%%%%%%%%%%%%%%%%%%%%%%%%%%

%\topic{P \large\hskip -1ex resentations}{~}
%
%\subtopic[2011]{Public Thesis Talk}{
%	Reed College, Portland, OR}
%
%\subtopic[2018]{Public thesis defense}{
%	San Jose State University, San Jose, CA}
%	
%\subtopic[2018]{AAPT Summer Conference}{
%	Presented a poster about thesis work on textbook design.}
% 
% 

 

  
\end{document}